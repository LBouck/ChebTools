%% Holzer's Golden Rule to presentations:
%% 1) Tell them what you're about to tell them
%% 2) Tell them
%% 3) Tell them what you just told them

\documentclass[xcolor={rgb,dvipsnames}]{beamer}		%% dvipsnames gives toooons of color options

\setbeamertemplate{navigation symbols}{}
\setbeamertemplate{footline}{}

% - Style is ornate.

\mode<presentation>							%% makes fullscreen on pdf reader and space bar to change slides
{
    \usetheme{Madrid}							%% specify theme
    %\usecolortheme{Lukestheme}						%% specify color scheme
    %\usefonttheme[onlysmall]{structuresmallcapsserif}
    \setbeamercovered{transparent}					%% makes items in enumerate transparent until you click on them
}

\usepackage{multimedia}							%% movies: .mp4 etc etc
\usepackage[english]{babel}						%% has all words/spellings/hyphenated words built in for whatever language you 							      					       specify in [ ]
\usepackage{amsfonts}
\usepackage{amsmath}
\usepackage{amsthm}
\usepackage{amssymb}
\usepackage{float}
\usepackage{subfigure}
\usepackage{graphics}
\usepackage{graphicx}
\usepackage{latexsym}
\usepackage{hyperref}
\usepackage{verbatim}
\usepackage{enumerate}
\usepackage{braket}
\usepackage{media9}
\usepackage{algorithm,algorithmic}
\usepackage{animate}
%\usepackage[noend]{algpseudocode}

\setbeamercovered{dynamic}
\graphicspath{ {/Users/Luke/Desktop/EXTREEMS/Write-ups} }
\addmediapath{{/Users/Luke/Desktop/EXTREEMS/Write-ups} }




%% Bibliography colors if you do not want bib items same color as the structure color

%% \setbeamercolor{bibliography entry author}{fg=Black}
%% \setbeamercolor{bibliography item}{fg=Black}
%% \setbeamercolor*{bibliography entry title}{fg=Black}
%% \setbeamercolor{bibliography entry journal}{fg=Black}
%% \setbeamercolor{bibliography entry note}{fg=Black}

\newcommand{\R}{\mathbb{R}}						%% example of a useful shortcut
\newcommand{\hs}{\hspace}
\newcommand{\vs}{\vspace}

%\usepackage{times}
%\usepackage[T1]{fontenc}

\title[Rootfinding with Chebyshev Polynomials] %optional
{Rootfinding with Chebyshev Polynomials in 2 Dimensions}

%\subtitle{}

\author[L. Bouck] % optional, will appear on every slide
{Lucas C. Bouck}	%% add more as needed

\institute[GMU] % optional, will appear on every slide
{ 
George Mason University\\
Department of Mathematical Sciences
}

\date[SURF]{August 2, 2017} 		%%% include conference name here, appears on every page
 


\begin{document}

\frame
{
  \titlepage
 \vspace*{-2ex}
    \begin{center}
    \structure{
    SURF Colloquium\\
    National Institute of Standards and Technology\\
     Boulder, CO}  \\[2ex]


{\small
Collaborator: Ian H. Bell (NIST Division 647)
}
\end{center}
}

\frame
{
\frametitle{Outline}
\tableofcontents					%% Here's where you give a brief overview as to what you are going to talk about 
}



%%%%%%%%%%%%%%%%%%%%%%%%%%

\section{The Problem}
\begin{frame}
\tableofcontents[currentsection]
\end{frame}
\frame{
\frametitle{The Problem}
\begin{block}{Global 2-D Rootfinding Problem}
We want to find all solutions to 
$$\begin{pmatrix} f(x,y) \\
g(x,y)
\end{pmatrix}
=\begin{pmatrix} 0 \\
0
\end{pmatrix}$$
in a given bounded domain $\Omega$
\end{block}

\begin{block}{The Big Picture: {\tt chebtools}}
\begin{itemize}
\item {\tt chebtools} is a C++ library for working with Chebyshev expansions developed by Ian Bell
\item Inspired by the Matlab library {\tt chebfun}
\item Solving this problem will be a significant feature for  {\tt chebtools}
\item With a higher level Python interface,  {\tt chebtools} could be useful for a wide range of users
\end{itemize}
\end{block}
}
\frame{
\frametitle{Example}
$f(x,y)=x^2-y^2$ and $g(x,y)=\sin\left(\pi\sqrt{x^2+y^2}\right)$ with the square domain $\Omega=[-1,1]^2$
\begin{center}
\begin{figure}
\includegraphics[trim=3cm 0cm 0cm 0cm,width=1\textwidth]{surface_plots1.pdf}
\end{figure}
\end{center}
}
\frame{
\frametitle{Example (cont.)}
Another view: We are finding where $f$ and $g$'s zero level sets intersect inside $\Omega$
\begin{center}
\begin{figure}
\includegraphics[trim=3cm 1cm 1cm 1cm,width=.9\textwidth]{level_sets1.pdf}
\end{figure}
\end{center}
}

\frame{
\frametitle{Local Methods May Not Be Good for Global rootfinding}
There are actually infinitely many roots outside $\Omega$ in our example. Local methods may converge to roots outside the domain of interest.
\begin{center}
\begin{figure}
\includegraphics[width=.7\textwidth]{newton_basins.pdf}
\caption{Initial guesses in white areas did not converge to roots in $\Omega$}
\end{figure}
\end{center}
}

\frame{
\frametitle{A Need for a Global Method}
Our example illustrates a few lessons
\begin{itemize}
\item Many iterative and local methods like Newton's method can only guarantee convergence to a root if the initial guess is sufficiently close
\item Sufficiently close depends on the problem
\item This may be extremely close for some roots like $(0,0)$ in our example
\end{itemize}
\bigskip
\bigskip
\begin{block}{Goal:}
To develop a truly global rootfinding method
\end{block}
}

\section{What are Chebyshev Polynomials and why use them?}
\begin{frame}
\tableofcontents[currentsection]
\end{frame}
\frame{
\frametitle{What are Chebyshev polynomials?}
%\noindent\fbox%
\hfill%
%\fbox
}

%\frame{
%\frametitle{What Chebyshev Polynomials Look Like}
%\begin{center}
%\begin{figure}
%\includegraphics[width=\textwidth]{cheb_poly_plot.pdf}
%\end{figure}
%\end{center}
%
%}

%\frame{
%\frametitle{Chebyshev polynomials provide accurate approximations}
%\begin{itemize}
%\item Lipschitz continuity $\implies$ uniform convergence of Chebyshev interpolations 
%\item $\nu-1$ continuous derivatives and a $\nu$th derivative of bounded variation $\implies$ Chebyshev interpolation converges like $\mathcal{O}(n^{-\nu})$
%\item Analytic function $\implies$ geometric convergence
%\end{itemize}
%\centering
%\begin{figure}
%\includegraphics[width=.65\textwidth]{cheb_convergence_plots.pdf}
%\end{figure}
%}

\frame{
\frametitle{Chebyshev polynomials are numerically stable}
\begin{itemize}
\item Interpolation on evenly spaced points is susceptible to the {\bf Runge phenomenon}
\item Chebyshev interpolation minimizes this effect.
%\item Our application area deals with Gaussians often, so minimizing the Runge phenomenon is useful
\end{itemize}
\begin{center}
\begin{figure}
\includegraphics[width=.9\textwidth]{runge_plot.pdf}
\caption{Interpolating $\frac{1}{1+25x^2}$}
\end{figure}
\end{center}
}

\section{The Algorithm}

\begin{frame}
\tableofcontents[currentsection]
\end{frame}



%\frame{
%\frametitle{Algorithm Overview}
%Given $f(x,y)$ and $g(x,y)$, we can find the roots with the following steps
%\begin{itemize}
%\item Interpolate $f,g$ with 2-D Chebyshev polynomials $p_f,p_g$
%\item Construct the Chebyshev-B\'{e}zout matrix polynomial $B$
%\item Linearize and solve a generalized eigenvalue problem to find the $x$ or $y$ values where $p_f,p_g$ have common roots
%\item Employ a 1-D rootfinding technique to find all the roots along the already found $x$ and $y$ values
%\end{itemize}
%}


%\frame{
%\frametitle{2-D Interpolation}
%The Idea behind 2-D interpolation is to continually interpolate the error from our previous approximations
%\begin{algorithm}[H]
%\begin{algorithmic}[1]
%\STATE $e_0(x,y)=f(x,y)$
%\STATE set tolerance $\varepsilon$
%\WHILE{$\max|e_k(x,y)|\geq\varepsilon$}
%\STATE find $(x_k,y_k)$ st $|e_k(x_k,y_k)|=\max|e_k(x,y)|$
%\STATE compute 1-D interpolation $\tilde{p}_{xk}(x,y_k)\approx e_k(x,y_k)$ 
%\STATE compute 1-D interpolation $\tilde{p}_{yk}(x_k,y)\approx e_k(x_k,y)/e_k(x_k,y_k)$
%\STATE $P_k(x,y) \gets P_{k-1}(x,y)+\tilde{p}_{xk}(x,y_k)\tilde{p}_{yk}(x_k,y)$ 
%\STATE $e_{k+1}(x,y) \gets e_k(x,y)-P_k(x,y)$ 
%\ENDWHILE
%\RETURN $P_k(x,y)$
%\end{algorithmic}
%\caption{Gaussian Elimination of Functions from Townsend (2014)}
%\label{2dinterp}
%\end{algorithm}
%}


%\begin{frame}{2-D Interpolation}
%    \centering
%    \animategraphics[loop,controls,height=.8\textheight]{1.3}{pivots_animation-}{0}{12}
%\end{frame}

\begin{frame}{2-D Interpolation}
\begin{block}{Idea:}
\begin{itemize}
\item Pick point of maximum error
\item Do 1-D interpolations in $x$ and $y$ directions
\end{itemize}
\end{block}
    \centering
    \animategraphics[loop,controls,height=.6\textheight]{1.3}{pivots2_animation-}{0}{28}
\end{frame}


%\frame{
%\frametitle{1st: 2-D Interpolation}
%%The pivots below are for a $64\times64$ grid of Chebyshev nodes. Note we do less interpolations than if we used all $64^2$ points
%The algorithm works by choosing a pivot position which is the maximum residual and doing 1-D interpolations in the $x$ and $y$ directions
%\begin{center}
%\begin{figure}
%\includegraphics[width=.9\textwidth]{contours_pivots.pdf}
%\caption{Plot of the contours of our example functions with pivots from the algorithm}
%\end{figure}
%\end{center}
%}

\frame{
\frametitle{Our Algorithm Can Provide Accurate Approximations}
\begin{itemize}
\item Almost machine precision accuracy with $x^2-y^2$
\item Struggles with  $\sin\left(\pi\sqrt{x^2+y^2}\right)$ due to the lack of differentiability at $(0,0)$
\end{itemize}
\begin{center}
\begin{figure}
\includegraphics[trim = 3cm 0cm 0cm 0.75cm, width=\textwidth]{pointwise_error_surface.pdf}
\end{figure}
\end{center}
}

%\frame{
%\frametitle{Comparison to chebfun2}
%\begin{center}
%\begin{table}[h!]
%\begin{tabular}{|c|c|c|}
%\hline
%Function & Our Code (64x64 points) & chebfun2 (adaptive)\\ \hline
% $x^2-y^2$ &106 ms & 153 ms \\ \hline
% $\sin\left(\pi\sqrt{x^2+y^2}\right)$ & 650 ms & 3222 ms \\ \hline
%\end{tabular}
%\caption{Speed Comparison}
%\end{table}
%\begin{table}[h!]
%\begin{tabular}{|c|c|c|}
%\hline
%Function & Our Code (64x64 points) & chebfun2 (adaptive)\\ \hline
% $x^2-y^2$ 					&6.89e-15  	& 5.55e-16 \\ \hline
% $\sin\left(\pi\sqrt{x^2+y^2}\right)$ 	& 2.97e-3  	& 2.98e-3 \\ \hline
%\end{tabular}
%\caption{Accuracy (Max Abs Error) Comparison}
%\end{table}
%\end{center}
%}
%\frame{
%\frametitle{Some Initial Differences with chebfun2}
%\begin{itemize}
%\item We do not have use an adaptive constructor while chebfun2 does
%	\begin{itemize}
%	\item This explains the slower speed of chebfun2 in the second example since $\sin(\pi\sqrt{x^2+y^2})$ is hard to approximate
%	\item An adaptive constructor does not make as much sense for our application
%	\end{itemize}
%\item We have note implemented a FFT version of Chebyshev Interpolation
%	\begin{itemize}
%	\item This could explain why chebfun2 is faster in the easy example
%	\item Implementing interpolation using FFT could further improvements our library
%	\end{itemize}
%\end{itemize}
%}
\frame{
\frametitle{B\'{e}zout Matrix Polynomials}
\begin{block}{We create a matrix polynomial $B(x)$ from our Chebyshevs}
\begin{itemize}
\item $B(x)$ is a matrix with polynomial entries
\item $\det\left(B(x)\right)=0$ $\iff$ common roots along the specific $x$ value
\item Solving $\det\left(B(x)\right)=0$ is a {\bf matrix polynomial eigenvalue problem}
\end{itemize}
\end{block}
\begin{center}
\begin{figure}
\includegraphics[height=.57\textheight]{bezout_det_plot.pdf}
\end{figure}
\end{center}
}

%\frame{
% We can solve the matrix polynomial eigenvalue problem by solving the generalized eigenvalue problem
%\begin{align*}
%\frac{1}{2}\begin{pmatrix}
%-B_{M-1} 	& I_n-A_{M-2} 	& -A_{M-3} 	& \cdots 	& -A_0	\\
%I_n	       	& 0		     	& I_n	       		&	    	&		\\
%		& \ddots		& \ddots		& \ddots	&		\\
%		&			& I_n			& 0		& I_n		\\
%		&			&			& 2I_n	& 0
%\end{pmatrix}&{\bf v}\\
%&=\lambda\begin{pmatrix}
%A_M	&	&		&	\\
%	&I_n	&		&	\\
%	&	&\ddots	&	\\
%	&	&		&I_n \\
%\end{pmatrix}{\bf v}
%\end{align*}
%}

\frame{
\frametitle{1-D Rootfinding and Local Refinement}
\begin{itemize}
\item We now have the possible $x$ values for where there are common roots
\item Employ a well known 1-D rootfinding method again using Chebyshev Polynomials
\item {\bf Problem:} Poor conditioning of matrix polynomial problem $\implies$ need for local refinement of roots
\item Refine the roots with Newton's method
\end{itemize}
}

\frame{
\frametitle{Results:}

}

%\frame{
%\frametitle{Similarities and Differences with chefun2}
%Similarities:
%\begin{itemize}
%\item The B\'{e}zout Resultant method for 2-D rootfinding and companion matrix methods for 1-D rootfinding
%\item Use Newton's method to polish/refine solutions
%\item Approximation methods are extremely similar
%\end{itemize}
%
%Differences:
%\begin{itemize}
%\item The goals of chebfun2 are much more general than just rootfinding, unlike our code
%\item We approximate with a set amount of points, unlike chebfun2, which is adaptive
%\item We 1-D root find $p_f+p_g$ and not $p_f$ and $p_g$ individually
%\item chefun2 subdivides to reduce the polynomial sizes, while the goal of our subdivision will be different
%\end{itemize}
%}

%\section{Current Ideas for Moving Beyond Rectangular Domains}
%\frame{
%\tableofcontents[currentsection]
%}
%\frame{
%\frametitle{Beyond Rectangular Domains}
%There are a few methods that could deal with non rectangular domains:
%\begin{itemize}
%\item Conformal maps: Can cause problem to be ill-conditioned
%\item Coordinate transformations: Need to know domain before-hand
%\item Method of frames
%\end{itemize}
%{\bf Goal:} We want a way to deal with any connected and sufficiently smooth domain
%}

%\frame{
%\frametitle{Beyond Rectangular Domains}
%{\bf Problem: }Chebyshev polynomials grow quickly outside $[-1,1]$. This could throw our approximation off and cause our algorithm to fail
%\begin{center}
%\begin{figure}
%\includegraphics[width=.7\textwidth]{cheb_growth.pdf}
%\caption{Even the 5th and 6th Chebyshev polynomials grow quickly outside $[-1,1]$}
%\end{figure}
%\end{center}
%
%}

%\frame{
%\frametitle{Chebyshev Interpolation Meets Constrained Least Squares}
%Using ideas from least squares, we have started to develop a method that does the following:
%\begin{itemize}
%\item Controls errors made by growth of Chebyshev polynomials
%\item Generalizes Chebyshev interpolation
%\item Can be done knowing currently available techniques such as SVD and Lagrange multipliers
%\end{itemize}
%
%Open questions:
%\begin{itemize}
%\item How does this technique affect the approximation made inside $[-1,1]$?
%\item Does this method guarantee convergence?
%\item Is it numerically stable?
%\end{itemize}
%}
%\frame{
%\frametitle{Chebyshev Interpolation Meets Constrained Least Squares}
%Typical Chebyshev interpolation involves a matrix vector product
%$$A{\bf f} = {\bf c}$$
%where $A$ is the interpolation matrix, ${\bf f}$ are the function values of $f(x)$ at the Chebyshev nodes and ${\bf c}$ are the Chebyshev coefficients. We could formulate the problem as solving
%
%$$A^{-1}{\bf c} = {\bf f}$$
% We still have interpolation if the above is a square system. To control the growth of the Chebyhshev polynomials outside $[-1,1]$, we can introduce a regularization term $\Gamma_\delta{\bf c}$ where 
% 
% $$\Gamma_\delta{\bf c} = \delta p(1+\delta) \implies \lVert\Gamma_\delta{\bf c}\rVert = \lvert \delta p(1+\delta)\rvert$$
% 
% where $p$ is our Chebyshev polynomial
% 
%}

%\frame{
%Instead of
%$$ A^{-1} = {\bf c}$$
%we now can reformulate our problem as 
%$$ \min_{\lVert\Gamma_\delta{\bf c}\rVert \leq \beta}\lVert A^{-1}{\bf c} - {\bf f}\rVert$$
%\begin{itemize}
%\item $\Gamma_\delta$ is controlling the error beyond the interval $[-1,1]$.
%\item $\delta=0\implies \Gamma_\delta\equiv0$ so our problem is a generalization of interpolation
%\item Framing the problem with a bound on $\lVert\Gamma_\delta{\bf c}\rVert$ provides a less ad hoc method than normal regularization  
%\item The new problem can still be solved with known techniques for constrained least squares problems such as SVD and Lagrange multipliers
%\end{itemize}
%}

\frame{
\frametitle{Conclusions and Future Work}
Key Contributions:
\begin{itemize}
\item Made significant progress in replicating the 2-D rootfinding algorithm in {\tt chebfun} with some modifications in C++11, which will be fully open source
\item Developed ideas and have a strategy for extending the algorithm to non-rectangular domains
\end{itemize}

Future Work:
\begin{itemize}
\item Further develop ideas on non-rectangular domains and subdivision strategies
\item Introduce GPU/parallel computing to the rootfinding process
\end{itemize}


Other Contributions while at NIST:
\begin{itemize}
\item Idea for using Gram-Schmidt process to create orthogonal terms for developing equations of state
\end{itemize}
}

\frame{
\frametitle{Acknowledgements}

\begin{itemize}
\item Dr.\ Ian Bell for the mentorship and guidance
\item Dr.\ Bradley Alpert for the lively discussions and brainstorming
\item The SURF program
\end{itemize}


}

\frame{
\frametitle{References}
 References here
}

\end{document}