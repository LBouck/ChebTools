%% Holzer's Golden Rule to presentations:
%% 1) Tell them what you're about to tell them
%% 2) Tell them
%% 3) Tell them what you just told them

\documentclass[xcolor={rgb,dvipsnames}]{beamer}		%% dvipsnames gives toooons of color options

\setbeamertemplate{navigation symbols}{}
\setbeamertemplate{footline}{}

% - Style is ornate.

\mode<presentation>							%% makes fullscreen on pdf reader and space bar to change slides
{
    \usetheme{Madrid}							%% specify theme
    %\usecolortheme{Lukestheme}						%% specify color scheme
    %\usefonttheme[onlysmall]{structuresmallcapsserif}
    \setbeamercovered{transparent}					%% makes items in enumerate transparent until you click on them
}

\usepackage{multimedia}							%% movies: .mp4 etc etc
\usepackage[english]{babel}						%% has all words/spellings/hyphenated words built in for whatever language you 							      					       specify in [ ]
\usepackage{amsfonts}
\usepackage{amsmath}
\usepackage{amsthm}
\usepackage{amssymb}
\usepackage{mathtools}
\usepackage{float}
\usepackage{subfigure}
\usepackage{graphics}
\usepackage{graphicx}
\usepackage{latexsym}
\usepackage{hyperref}
\usepackage{verbatim}
\usepackage{enumerate}
\usepackage{braket}
\usepackage{media9}
\usepackage{algorithm,algorithmic}
\usepackage{animate}
%\usepackage[noend]{algpseudocode}

\setbeamercovered{dynamic}
\graphicspath{ {/Users/Luke/Desktop/EXTREEMS/Write-ups} }
\addmediapath{{/Users/Luke/Desktop/EXTREEMS/Write-ups} }




%% Bibliography colors if you do not want bib items same color as the structure color

%% \setbeamercolor{bibliography entry author}{fg=Black}
%% \setbeamercolor{bibliography item}{fg=Black}
%% \setbeamercolor*{bibliography entry title}{fg=Black}
%% \setbeamercolor{bibliography entry journal}{fg=Black}
%% \setbeamercolor{bibliography entry note}{fg=Black}

\newcommand{\R}{\mathbb{R}}						%% example of a useful shortcut
\newcommand{\hs}{\hspace}
\newcommand{\vs}{\vspace}

%\usepackage{times}
%\usepackage[T1]{fontenc}

\title[Root Finding with Chebyshev Polynomials] %optional
{Root Finding with Chebyshev Polynomials in 2 Dimensions}

%\subtitle{}

\author[L. Bouck] % optional, will appear on every slide
{Lucas C. Bouck}	%% add more as needed

\institute[GMU] % optional, will appear on every slide
{ 
George Mason University\\
Department of Mathematical Sciences
}

\date[ACMS]{September 8, 2017} 		%%% include conference name here, appears on every page
 


\begin{document}

\frame
{
  \titlepage
 \vspace*{-2ex}
    \begin{center}
    \structure{
    Applied and Computational Math Seminar\\
    George Mason University\\
     Fairfax, VA}  \\[2ex]


{\small
Collaborator: Ian H. Bell (NIST Applied Chemicals and Materials Division)
}
\end{center}
}

\frame
{
\frametitle{Outline}
\tableofcontents					%% Here's where you give a brief overview as to what you are going to talk about 
}



%%%%%%%%%%%%%%%%%%%%%%%%%%

\section{Summary of the Problem and Background}
\begin{frame}
\tableofcontents[currentsection]
\end{frame}
\frame{
\frametitle{The Problem}
\begin{block}{Global 2-D Root Finding Problem}
We want to find all solutions ${\bf x}\in\Omega\subset\R^2$ to 
$${\bf F}({\bf x})={\bf 0}$$
\end{block}

Additional assumptions
\begin{itemize}
\item $\Omega$ will be a bounded domain and rectangular for now
\item Treat ${\bf F}$ as $\begin{pmatrix} f(x,y) \\
g(x,y)
\end{pmatrix}$, where $x,y\in\R$ and $f,g$ are scalar functions
\item Assume $f,g$ are smooth enough (more on this later)
\item Assume finitely many roots
\end{itemize}
}

\frame{
\frametitle{Background and Motivation}
The Big Picture: {\tt chebtools}
\begin{itemize}
\item {\tt chebtools} is a C++ library for working with Chebyshev expansions developed by Ian Bell
\item Inspired by the Matlab library {\tt chebfun}
\item 2-D root finding will become a central feature of {\tt chebtools}
\end{itemize}

Applications:
\begin{itemize}
\item One example is determining thermodynamic properties of steam or water (Kunick, Kretzschmar, and Gampe 2008)
\item With a higher level Python interface, {\tt chebtools} could be useful for a wide range of users
\end{itemize}

}



\frame{
\frametitle{Example}
$f(x,y)=x^2-y^2$ and $g(x,y)=\sin\left(\pi\sqrt{x^2+y^2}\right)$ with the square domain $\Omega=[-1,1]^2$
\begin{center}
\begin{figure}
\includegraphics[trim=3cm 0cm 0cm 0cm,width=1\textwidth]{surface_plots1.pdf}
\end{figure}
\end{center}
}



%\frame{
%\frametitle{Example (cont.)}
%Another view: We are finding where $f$ and $g$'s zero level sets intersect inside %$\Omega$
%\begin{center}
%\begin{figure}
%\includegraphics[trim=3cm 1cm 1cm 1cm,width=.9\textwidth]{level_sets1.pdf}
%\end{figure}
%\end{center}
%}

\frame{
\frametitle{A Need for a Method to Find All Roots}
\begin{itemize}
\item Iterative methods may converge to roots outside the domain of interest.
\item Globalized Newton methods guarantee convergence to a root but only find them one at a time (Deuflhard, 2011)
\item We need a method to find all solutions at once
\end{itemize}
\begin{center}
\begin{figure}
\includegraphics[width=.5\textwidth]{newton_basins.pdf}
\caption{Initial guesses in white areas did not converge to roots in $\Omega$}
\end{figure}
\end{center}
}

%\frame{
%\frametitle{A Need For a Global Method}
%Our example illustrates a few lessons
%\begin{itemize}
%\item Many iterative and local methods like Newton's method can only guarantee convergence to a root if the initial guess is sufficiently close
%\item Sufficiently close depends on the problem
%\item Our example is an example of this problem
%\item Other Globalized Newton methods can guarantee convergence to a root but only find them one at a time (Deuflhard, 2011)
%\end{itemize}
%\bigskip
%\bigskip
%\begin{block}{Goal:}
%To develop a global root finding method to find all solutions at once
%\end{block}
%}

\section{What Are Chebyshev Polynomials and Why Use Them?}
\begin{frame}
\tableofcontents[currentsection]
\end{frame}


\frame{
\frametitle{What are Chebyshev Polynomials?}
%\noindent\fbox{%
\begin{block}{Chebyshev polynomials}
$$T_n(x)=\cos\left(n\arccos(x)\right),\, x\in[-1,1]$$
\end{block}
\begin{minipage}{.4\linewidth}
%Orthogonal under a weighted inner product:
%$$\int_{-1}^{1} T_n(x)T_k(x)w(x)\, dx=0$$ when $n\neq k$ 
%\\ and $w(x)=\frac{1}{\sqrt{1-x^2}}$
%\\
\begin{itemize}
\item Orthogonal under an inner product with weight $w(x)=\frac{1}{\sqrt{1-x^2}}$
\item Lipschitz continuity $\implies$ uniform convergence of Chebyshev interpolations
 \item Analytic function $\implies$ geometric convergence
\end{itemize}

%Numerically stable for interpolating functions
\end{minipage}%}%
\hfill%
%\fbox
}

%\frame{
%\frametitle{What Chebyshev Polynomials Look Like}
%\begin{center}
%\begin{figure}
%\includegraphics[width=\textwidth]{cheb_poly_plot.pdf}
%\end{figure}
%\end{center}
%
%}

%\frame{
%\frametitle{Chebyshev Interpolation Converges Quickly}
%\begin{itemize}
%\item Lipschitz continuity $\implies$ uniform convergence of Chebyshev interpolations 
%\item $\nu-1$ continuous derivatives and a $\nu$th derivative of bounded variation $\implies$ Chebyshev interpolation converges like $\mathcal{O}(n^{-\nu})$
%\item Analytic function $\implies$ geometric convergence
%\item Same results will apply for 2-D approximations
%\end{itemize}
%\centering
%\begin{figure}
%\includegraphics[width=.5\textwidth]{cheb_convergence_plots.pdf}
%\end{figure}
%}

%\frame{
%\frametitle{Chebyshev Interpolation Avoids the Runge Phenomenon}
%\begin{itemize}
%\item Interpolation on evenly spaced points is susceptible to the {\bf Runge phenomenon}
%\item Chebyshev interpolation minimizes this effect.
%\end{itemize}
%\begin{center}
%\begin{figure}
%\includegraphics[width=.9\textwidth]{runge_plot.pdf}
%\caption{Interpolating $\frac{1}{1+25x^2}$}
%\end{figure}
%\end{center}
%}

\section{The Algorithm for Rectangular Domains from {\tt chebfun}}

\begin{frame}
\tableofcontents[currentsection]
\end{frame}



%\frame{
%\frametitle{Algorithm Overview}
%The bulk of the algorithm is a replication of the algorithm introduced by Alex Townsend in {\tt chebfun}
%\begin{block}{The Algorithm}
%Given $f(x,y)$ and $g(x,y)$, we can find the roots with the following steps
%\begin{itemize}
%\item Interpolate $f,g$ with 2-D Chebyshev polynomials $p_f,p_g$
%\item Construct the Chebyshev-B\'{e}zout matrix polynomial $B$
%\item Linearize and solve a generalized eigenvalue problem to find the $x$ or $y$ values where $p_f,p_g$ have common roots
%\item Employ a 1-D rootfinding technique to find all the roots along the already found $x$ and $y$ values
%\end{itemize}
%\end{block}
%}


%\frame{
%\frametitle{2-D Interpolation}
%The Idea behind 2-D interpolation is to continually interpolate the error from our previous approximations
%\begin{algorithm}[H]
%\begin{algorithmic}[1]
%\STATE $e_0(x,y)=f(x,y)$
%\STATE set tolerance $\varepsilon$
%\WHILE{$\max|e_k(x,y)|\geq\varepsilon$}
%\STATE find $(x_k,y_k)$ st $|e_k(x_k,y_k)|=\max|e_k(x,y)|$
%\STATE compute 1-D interpolation $\tilde{p}_{xk}(x,y_k)\approx e_k(x,y_k)$ 
%\STATE compute 1-D interpolation $\tilde{p}_{yk}(x_k,y)\approx e_k(x_k,y)/e_k(x_k,y_k)$
%\STATE $P_k(x,y) \gets P_{k-1}(x,y)+\tilde{p}_{xk}(x,y_k)\tilde{p}_{yk}(x_k,y)$ 
%\STATE $e_{k+1}(x,y) \gets e_k(x,y)-P_k(x,y)$ 
%\ENDWHILE
%\RETURN $P_k(x,y)$
%\end{algorithmic}
%\caption{Gaussian Elimination of Functions from Townsend (2014)}
%\label{2dinterp}
%\end{algorithm}
%}


%\begin{frame}{2-D Interpolation}
%    \centering
%    \animategraphics[loop,controls,height=.8\textheight]{1.3}{pivots_animation-}{0}{12}
%\end{frame}

\begin{frame}{2-D Interpolation}
\begin{itemize}
\item Define our first error function as $e_0(x,y)=f(x,y)$ 
\item Define later error functions as $e_k(x,y)\coloneqq e_{k-1}(x,y)-P_{k-1}(x,y)$ where $P_{k-1}$ is our approximation
\end{itemize}
\begin{block}{Algorithm Outline}
\begin{itemize}
\item Find $(x_k,y_k)$ s.t.  $|e_k(x_k,y_k)|=\max|e_k(x,y)|$
\item Do 1-D interpolations of $e_k(x,y_k)$ and $e_k(x_k,y)/e_k(x_k,y_k)$ denoted $p_x(x)$ and $p_y(y)$
\item Compute new approximation $P_{k}(x,y) = P_{k-1}(x,y)+ p_x(x)p_y(y)$
\end{itemize}
\end{block}
\end{frame}

\begin{frame}{Interpolation From Our Example}
    \centering
    \animategraphics[loop,controls,height=.8\textheight]{1.3}{pivots2_animation-}{0}{28}
\end{frame}


%\frame{
%\frametitle{1st: 2-D Interpolation}
%%The pivots below are for a $64\times64$ grid of Chebyshev nodes. Note we do less interpolations than if we used all $64^2$ points
%The algorithm works by choosing a pivot position which is the maximum residual and doing 1-D interpolations in the $x$ and $y$ directions
%\begin{center}
%\begin{figure}
%\includegraphics[width=.9\textwidth]{contours_pivots.pdf}
%\caption{Plot of the contours of our example functions with pivots from the algorithm}
%\end{figure}
%\end{center}
%}

%\frame{
%\frametitle{Our Algorithm Can Provide Accurate Approximations}
%\begin{itemize}
%\item Almost machine precision accuracy with $x^2-y^2$
%\item Struggles with  $\sin\left(\pi\sqrt{x^2+y^2}\right)$ due to the lack of %differentiability at $(0,0)$
%\end{itemize}
%\begin{center}
%\begin{figure}
%\includegraphics[trim = 3cm 0cm 0cm 0.75cm, width=\textwidth]%{pointwise_error_surface.pdf}
%\end{figure}
%\end{center}
%}

%\frame{
%\frametitle{Comparison to chebfun2}
%\begin{center}
%\begin{table}[h!]
%\begin{tabular}{|c|c|c|}
%\hline
%Function & Our Code (64x64 points) & chebfun2 (adaptive)\\ \hline
% $x^2-y^2$ &106 ms & 153 ms \\ \hline
% $\sin\left(\pi\sqrt{x^2+y^2}\right)$ & 650 ms & 3222 ms \\ \hline
%\end{tabular}
%\caption{Speed Comparison}
%\end{table}
%\begin{table}[h!]
%\begin{tabular}{|c|c|c|}
%\hline
%Function & Our Code (64x64 points) & chebfun2 (adaptive)\\ \hline
% $x^2-y^2$ 					&6.89e-15  	& 5.55e-16 \\ \hline
% $\sin\left(\pi\sqrt{x^2+y^2}\right)$ 	& 2.97e-3  	& 2.98e-3 \\ \hline
%\end{tabular}
%\caption{Accuracy (Max Abs Error) Comparison}
%\end{table}
%\end{center}
%}
%\frame{
%\frametitle{Some Initial Differences with chebfun2}
%\begin{itemize}
%\item We do not have use an adaptive constructor while chebfun2 does
%	\begin{itemize}
%	\item This explains the slower speed of chebfun2 in the second example since $\sin(\pi\sqrt{x^2+y^2})$ is hard to approximate
%	\item An adaptive constructor does not make as much sense for our application
%	\end{itemize}
%\item We have note implemented a FFT version of Chebyshev Interpolation
%	\begin{itemize}
%	\item This could explain why chebfun2 is faster in the easy example
%	\item Implementing interpolation using FFT could further improvements our library
%	\end{itemize}
%\end{itemize}
%}



%\frame{
%\frametitle{B\'{e}zout Matrix Polynomials}
%\begin{block}{We create a matrix polynomial $B(x)$ from our Chebyshevs}
%\begin{itemize}
%\item $B(x)$ is a matrix with polynomial entries
%\item $\det\left(B(x)\right)=0$ $\iff$ common roots along the specific $x$ value
%\item Solving $\det\left(B(x)\right)=0$ is a {\bf matrix polynomial eigenvalue problem}
%\end{itemize}
%\end{block}
%\begin{center}
%\begin{figure}
%\includegraphics[height=.57\textheight]{bezout_det_plot.pdf}
%\end{figure}
%\end{center}
%}

\frame{
\frametitle{B\'{e}zout Resultant Method}
\begin{itemize}
\item From our Chebyshev approximations $p_f(x,y),p_g(x,y)$, we can construct a B\'{e}zout matrix polynomial $B(x)$
\item $\det\left(B(x_0)\right)=0$ $\iff$ $p_f(x_0,\cdot)$ and $p_g(x_0,\cdot)$ have common root
\end{itemize}

If $p_f,p_g$ are of degree $(m_f,n_f),(m_g,n_g)$ respectively, then the resulting form of the matrix polynomial of degree $M$ is
$$B(x) = \sum_{i=0}^{M}B_iT_i(x)$$
\begin{itemize}
\item $B_i$ are square matrices of size $n = \max(n_f,n_g)$ and $M\leq m_f+m_g$.
\item Solving $\det\left(B(x_0)\right)=0$ involves linearizing $B(x)$ (Nakatsukasa et. al. 2016)
\item Finally solve a generalized eigenvalue problem with computational complexity of $\mathcal{O}\left(M^3n^3\right)$
\end{itemize}
}

%\begin{frame}{Solving the Matrix Polynomial Eigenvalue Problem}
%Expressing $B(x) = \sum_{i=0}^{M}B_iT_i(x)$, then we can solve $\det\left(B(x)\right)=0$ by solving the generalized eigenvalue problem
%\begin{align*}
%\frac{1}{2}&\begin{pmatrix}
%-B_{M-1} 	& I_n-B_{M-2} 	& -B_{M-3} 	& \cdots 	& -B_0	\\
%I_n	       	& 0		     	& I_n	       		&	    	&		\\
%		& \ddots		& \ddots		& \ddots	&		\\
%		&			& I_n			& 0		& I_n		\\
%		&			&			& 2I_n	& 0
%\end{pmatrix}{\bf v}\\
%&=\lambda\begin{pmatrix}
%B_M	&	&		&	\\
%	&I_n	&		&	\\
%	&	&\ddots	&	\\
%	&	&		&I_n \\
%\end{pmatrix}{\bf v}
%\end{align*}
%This is the most computationally intensive part of the algorithm and has computational complexity of $\mathcal{O}\left(M^3n^3\right)$
%\end{frame}

\begin{frame}{Conditioning of the B\'{e}zout Matrix Polynomials}
\begin{itemize}
\item The B\'{e}zout matrix polynomial technique can square the condition number of the common root.
\item The {\tt chebfun} algorithm refines the roots by recomputing the matrix polynomial problem on a zoomed in region of the root
\end{itemize}


\begin{center}
\begin{figure}
\includegraphics[height=.63\textheight]{bezout_det_plot.pdf}
\end{figure}
\end{center}
\end{frame}

%\frame{
% We can solve the matrix polynomial eigenvalue problem by solving the generalized eigenvalue problem
%\begin{align*}
%\frac{1}{2}\begin{pmatrix}
%-B_{M-1} 	& I_n-A_{M-2} 	& -A_{M-3} 	& \cdots 	& -A_0	\\
%I_n	       	& 0		     	& I_n	       		&	    	&		\\
%		& \ddots		& \ddots		& \ddots	&		\\
%		&			& I_n			& 0		& I_n		\\
%		&			&			& 2I_n	& 0
%\end{pmatrix}&{\bf v}\\
%&=\lambda\begin{pmatrix}
%A_M	&	&		&	\\
%	&I_n	&		&	\\
%	&	&\ddots	&	\\
%	&	&		&I_n \\
%\end{pmatrix}{\bf v}
%\end{align*}
%}

\frame{
\frametitle{1-D Root Finding}
\noindent
\begin{minipage}{.4\linewidth}
\begin{block}{1-D Root Finding}
\begin{itemize}
\item We now have the possible $x$ values for where there are common roots
\item Employ a companion matrix method finding the roots of 1-D Chebyshev polynomials
\end{itemize}
\end{block}
\end{minipage}%
\hspace{.01\linewidth}
\noindent
\begin{minipage}{.53\linewidth}
\begin{center}
\begin{figure}
\includegraphics[width=\linewidth]{cheb1d_roots.pdf}
\end{figure}
\end{center}
\end{minipage}%
%\begin{block}{Problem: Poor conditioning of matrix polynomial problem}
%\begin{itemize}
%\item  Need for local refinement of roots
%\item Refine the roots with Newton's method
%\end{itemize}
%\end{block}

}
\section{Results and Future Work}
\frame{
\tableofcontents[currentsection]
}
\frame{
\frametitle{First Results}


\begin{tabular}{ |c|c|c|c|c|  }
 \hline
 & \multicolumn{2}{|c|}{{\tt chebtools}}&\multicolumn{2}{|c|}{{\tt chebfun}} \\
 \hline
 Functions							& 2 Norm Error 	& Time (s)	& 2 Norm Error	& Time (s)\\
 \hline
 $F_1(x,y),F_2(x,y)$ & $6.97\times10^{-32}$&0.004268 &$7.7\times10^{-16}$ &0.522 \\
 \hline
 $G_1(x,y),G_2(x,y)$ & $6.21\times10^{-16}$ & 214.059 &$2.92\times10^{-10}$ &0.294 \\
 \hline
 $H_1(x,y),H_2(x,y)$ &$3.24\times10^{-16}$ & 195.903 &$2.77\times10^{-11}$ &0.296 \\
 \hline
\end{tabular}
\begin{align*}
F_1(x,y) &= T_3(x)-13T_1(x) & F_2(x,y) &= T_3(y)-13T_1(y)\\
G_1(x,y) &= \cos(\pi x)(y-2) & G_2(x,y) &= (y-.9)(x-2)\\
H_1(x,y) &= \cos\left(\pi x-\frac{\pi}{10}\right)(y-2) & H_2(x,y) &= (y-.1)(y-.9)(x-2)
\end{align*}
\begin{itemize}
\item {\tt chebfun} is much faster than {\tt chebtools}
\item {\tt chebfun} has a domain subdivision strategy for reducing the size of the generalized eigenvalue problem
\end{itemize}
}


\frame{
\frametitle{Application Motivates Future Work}
\begin{itemize}
\item Empirical equations of state in thermodynamics have ranges of validity which are not rectangular domains
\item Solutions outside the domain will not make physical sense
\item Some properties are not defined at certain points (ex: Critical Point)
%\item Current work for us involves implementing a new subdivision strategy
%\item The goal of our strategy will be to deal with non-rectangular domains
%\item The need for dealing with non-rectangular domains comes from our application of interest in thermodynamic calculations
\end{itemize}
%insert plot of water with critical region

\begin{center}
\begin{figure}
\includegraphics[width=.7\linewidth]{hp_water.pdf}
\end{figure}
\end{center}
}

\frame{
\frametitle{Current Work for Moving Beyond Rectangular Domains}
Suppose we have
\begin{itemize}
\item Subdivided such that part of the boundary of the subdomain of interest, $\Omega_s$, can be expressed as a function 
\item mapped the rectangle containing $\Omega_s$ has been mapped to a reference square $[-1,1]^2$
\end{itemize}
%add plot of subdivided domain

\begin{block}{Our Idea:}
\begin{itemize}
\item Same 2-D interpolation procedure as with the rectangular domain
\item Instead of 1-D interpolations, solve a least squares fitting problem with nodes inside $\Omega_s$
\item Constrain the least squares solution s.t. $p(\pm 1)\leq B$ for some bound $B$
\end{itemize}
\end{block}
}

\frame{
\frametitle{Future Work}
For Non-Rectangular Domains
\begin{itemize}
\item Further develop ideas for Chebyshev approximation methods
\item Provide analysis of the new method of approximations
\end{itemize}



Other Future Work
\begin{itemize}
\item Introduce GPU/parallel computing to the root finding process
\end{itemize}
}




%\frame{
%\frametitle{Back to Our First Example}
%\begin{block}{{\tt chebfun} also struggled}
%\begin{itemize}
%\item Found 4 of the roots with an error of $5.55\times10^{-16}$
%\item Took 208.8 seconds
%\item Missed the root at $(0,0)$
%\end{itemize}
%\end{block}
%\begin{center}
%\begin{figure}
%\includegraphics[trim=3cm 0cm 0cm 0cm,width=.8\textwidth]{surface_plots1.pdf}
%\end{figure}
%\end{center}
%
%}


%\frame{
%\frametitle{Similarities and Differences with chefun2}
%Similarities:
%\begin{itemize}
%\item The B\'{e}zout Resultant method for 2-D rootfinding and companion matrix methods for 1-D rootfinding
%\item Use Newton's method to polish/refine solutions
%\item Approximation methods are extremely similar
%\end{itemize}
%
%Differences:
%\begin{itemize}
%\item The goals of chebfun2 are much more general than just rootfinding, unlike our code
%\item We approximate with a set amount of points, unlike chebfun2, which is adaptive
%\item We 1-D root find $p_f+p_g$ and not $p_f$ and $p_g$ individually
%\item chefun2 subdivides to reduce the polynomial sizes, while the goal of our subdivision will be different
%\end{itemize}
%}

%\section{Current Ideas for Moving Beyond Rectangular Domains}
%\frame{
%\tableofcontents[currentsection]
%}
%\frame{
%\frametitle{Beyond Rectangular Domains}
%There are a few methods that could deal with non rectangular domains:
%\begin{itemize}
%\item Conformal maps: Can cause problem to be ill-conditioned
%\item Coordinate transformations: Need to know domain before-hand
%\item Method of frames
%\end{itemize}
%{\bf Goal:} We want a way to deal with any connected and sufficiently smooth domain
%}

%\frame{
%\frametitle{Beyond Rectangular Domains}
%{\bf Problem: }Chebyshev polynomials grow quickly outside $[-1,1]$. This could throw our approximation off and cause our algorithm to fail
%\begin{center}
%\begin{figure}
%\includegraphics[width=.7\textwidth]{cheb_growth.pdf}
%\caption{Even the 5th and 6th Chebyshev polynomials grow quickly outside $[-1,1]$}
%\end{figure}
%\end{center}
%
%}

%\frame{
%\frametitle{Chebyshev Interpolation Meets Constrained Least Squares}
%Using ideas from least squares, we have started to develop a method that does the following:
%\begin{itemize}
%\item Controls errors made by growth of Chebyshev polynomials
%\item Generalizes Chebyshev interpolation
%\item Can be done knowing currently available techniques such as SVD and Lagrange multipliers
%\end{itemize}
%
%Open questions:
%\begin{itemize}
%\item How does this technique affect the approximation made inside $[-1,1]$?
%\item Does this method guarantee convergence?
%\item Is it numerically stable?
%\end{itemize}
%}
%\frame{
%\frametitle{Chebyshev Interpolation Meets Constrained Least Squares}
%Typical Chebyshev interpolation involves a matrix vector product
%$$A{\bf f} = {\bf c}$$
%where $A$ is the interpolation matrix, ${\bf f}$ are the function values of $f(x)$ at the Chebyshev nodes and ${\bf c}$ are the Chebyshev coefficients. We could formulate the problem as solving
%
%$$A^{-1}{\bf c} = {\bf f}$$
% We still have interpolation if the above is a square system. To control the growth of the Chebyhshev polynomials outside $[-1,1]$, we can introduce a regularization term $\Gamma_\delta{\bf c}$ where 
% 
% $$\Gamma_\delta{\bf c} = \delta p(1+\delta) \implies \lVert\Gamma_\delta{\bf c}\rVert = \lvert \delta p(1+\delta)\rvert$$
% 
% where $p$ is our Chebyshev polynomial
% 
%}

%\frame{
%Instead of
%$$ A^{-1} = {\bf c}$$
%we now can reformulate our problem as 
%$$ \min_{\lVert\Gamma_\delta{\bf c}\rVert \leq \beta}\lVert A^{-1}{\bf c} - {\bf f}\rVert$$
%\begin{itemize}
%\item $\Gamma_\delta$ is controlling the error beyond the interval $[-1,1]$.
%\item $\delta=0\implies \Gamma_\delta\equiv0$ so our problem is a generalization of interpolation
%\item Framing the problem with a bound on $\lVert\Gamma_\delta{\bf c}\rVert$ provides a less ad hoc method than normal regularization  
%\item The new problem can still be solved with known techniques for constrained least squares problems such as SVD and Lagrange multipliers
%\end{itemize}
%}

%\frame{
%\frametitle{Conclusions and Future Work}
%Key Contributions:
%\begin{itemize}
%\item Made significant progress in replicating the 2-D rootfinding algorithm in {\tt chebfun} with some modifications in C++11, which will be fully open source
%\item Developed ideas and have a strategy for extending the algorithm to non-rectangular domains
%\end{itemize}
%
%Future Work:
%\begin{itemize}
%\item Further develop ideas on non-rectangular domains and subdivision strategies
%\item Introduce GPU/parallel computing to the rootfinding process
%\end{itemize}
%
%
%Other Contributions while at NIST:
%\begin{itemize}
%\item Idea for using Gram-Schmidt process to create orthogonal terms for developing equations of state
%\end{itemize}
%}

\frame{
\frametitle{Conclusions and Acknowledgements}
Contributions:
\begin{itemize}
\item Made significant progress in replicating the 2-D root finding algorithm in {\tt chebfun} with some modifications in C++11
\item Began to develop ideas for extending the algorithm to non-rectangular domains
\end{itemize}

Acknowledgments
\begin{itemize}
\item Ian Bell and Bradley Alpert (NIST Applied Math Division)
\item The Summer Undergraduate Research Fellowship program at NIST
\end{itemize}
\bigskip
\bigskip
\begin{center}
{\large Questions?}
\end{center}
}

\frame{
\frametitle{References}
{\scriptsize
$[1]$ R. Bartels and G. Stewart. {\it Solution of the Matrix Equation} $AX+XB=C$: {\it Algorithm 432}, Comm. ACM, 1972.\\
$[2]$ I. Bell et. al. {\it Pure and Pseudo-pure Fluid Thermophysical Property Evaluation and
             the Open-Source Thermophysical Property Library CoolProp}, Industrial \& Engineering Chemical Research, 2014.\\
$[3]$ P. Deuflhard. {\it Newton Methods for Nonlinear Problems}, Springer, Berlin, 2011.\\
$[4]$ G. Golub and C. Van Loan. {\it Matrix Computations}. The Johns Hopkins University Press, Baltimore, 2013.\\
$[5]$ M. Kunick et. al. {\it Fast Calculation of Thermodynamic Properties of Water and Steam in Process Modelling using Spline Interpolation}. 2008.\\
$[6]$ Y. Nakatsukasa et. al. {\it Vector Spaces of Linearizations for Matrix Polynomials: A Bivariate Polynomial Approach}, arXiv:1610.01859v1, 2016.\\
 $[7]$ A. Townsend. {\it Computing with Functions in Two Dimensions} (Doctoral Dissertation). Oxford University, 2014.\\
 $[8]$ L. N. Trefethen, {\it Approximation Theory and Approximation Practice}, SIAM, Philadelphia, 2013.\\}
}

\end{document}